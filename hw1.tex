%{
\documentclass[11pt, twoside]{article}   	% use "amsart" instead of "article" for AMSLaTeX format
\usepackage{geometry}                		% See geometry.pdf to learn the layout options. There are lots.
\geometry{letterpaper}                   		% ... or a4paper or a5paper or ... 
%\geometry{landscape}                		% Activate for for rotated page geometry
\usepackage[parfill]{parskip}    		% Activate to begin paragraphs with an empty line rather than an indent
\usepackage{graphicx}				% Use pdf, png, jpg, or eps§ with pdflatex; use eps in DVI mode
								% TeX will automatically convert eps --> pdf in pdflatex	
\usepackage{amssymb}
\usepackage{color}
\usepackage{matlab-prettifier}
\usepackage{verbatim}
\usepackage{fancyvrb}
\usepackage{multicol}
\usepackage{bm}
\lstset{style=Matlab-editor,basicstyle=\ttfamily}

\sloppy
\definecolor{lightgray}{gray}{0.5}
\newenvironment{matlab}{\comment}{\endcomment}
\newenvironment{matlabv}{\lstlisting}{\endlstlisting}	


\title{Homework \# 1}
\author{Cody W. Eilar}
%\date{}							% Activate to display a given date or no date

\begin{document}
\maketitle
\section{Introduction} 
This paper investigates support vector machines (SVM) for both classification 
and regression of simple datasets. It explores these methods by leveraging
the \textbf{MATLAB} bindings for \textbf{LIBSVM}. 

\begin{matlab}
%}
close all; 
clear all;
% Set the random seed for reproducibility 
rand('seed', 1e5);
randn('seed', 1e5);

fid = fopen('tmp/computer.tex', 'w'); 
fprintf(fid, [computer]); 
fclose(fid); 

fid = fopen('tmp/matlabver.tex', 'w'); 
a = ver('matlab'); 
fprintf(fid, [a.Name ' version ' a.Version]); 
fclose(fid); 
%{
\end{matlab}

\section{Methods}
 In this section we look at how we can setup a scenario 
for testing how well a support vector machine works. In order to do this, 
we need to first create some data. For the simple linear case, we can create
four Gaussian distributions and plot them. Also, for reproducibiity 
we note that the code was run using \input{tmp/computer} running
 \input{tmp/matlabver} . That said, we can create four Gaussians
with the following code and classify them as either being positive 1 or negative 1. 

\begin{matlab}
%}
close all; 
clear all;
% Set the random seed for reproducibility 
rand('seed', 1e5);
randn('seed', 1e5);

fid = fopen('tmp/computer.tex', 'w'); 
fprintf(fid, computer); 
fclose(fid); 
%{
\end{matlab}

\begin{lstlisting}
%}
sigma = 0.2; 
N = 10; 
[X, Y] = plot2d(sigma, N);
title('Four Gaussian distributions of points'); 
saveas(gcf,'tmp/gaussian.eps','epsc')
%{
\end{lstlisting}  


\begin{figure}[h]
\centering
\includegraphics[width=4in]{tmp/gaussian.eps}
\caption{Gaussian plot}
\label{fig:gaus} 
\end{figure}

From Figure  \ref{fig:gaus}, we can see that there is a natural separation between the two labels. 
The code shows that the first half of the  variable is negative 1, and the second half is positive 1. 
We now have a simple linear problem to classify. We can further say that the classes are almost
linearly separable, but there is some overlap. So even if we end up with a great classifier, there
will still be points that are not properly classified. 

The next step is to now classify the data using \textbf{LIBSVM}. This is done by calling \texttt{svmtrain}.
 Typically, we would want to break up the training and testing data, but since 
this is just a simple example, we will using the training data for the testing as well. The following code 
trains the model: 
\begin{lstlisting}
%}
delete 'tmp/svmtrain.txt'
diary 'tmp/svmtrain.txt'
model = svmtrain(Y', X, '-s 0 -t 0 -c 100');
diary off; 
%{
\end{lstlisting}

\color{lightgray}

\VerbatimInput{tmp/svmtrain.txt}

\color{black}

The parameters mean the following: 
\begin{itemize}
\item -s 0 - This defines the kernel type
\item -t 0 - This specifies the kernel type, 0 is for a linear kernel
\item -c 100 - This specifies the cost, here the cost is 100
\end{itemize}

Now that we have the model, we need to see how well we can classify
the data using \texttt{svmpredict}. 
\begin{lstlisting}
%}
delete 'tmp/svmpredict.txt'
diary 'tmp/svmpredict.txt'
[predicted_label, accuracy, prob_estimates] = svmpredict(Y', X, model); 
diary off
%{
\end{lstlisting}

The output from running \texttt{svmpredict} is the following: 
\color{lightgray}
\VerbatimInput{tmp/svmpredict.txt}
\color{black}

It is obvious from these results that we can predict the data pretty well. 
but what happens if we compare this to a different method? Will we get
similar results? An easy test is to classify the data using \textbf{MATLAB}
\'s function \texttt{lsqlin} with the same parameters that we used 
for the support vector machine. 

Before we compare though, let us start by plotting the shattering
hyperplane along with circling the support vectors. 
\begin{matlab}
%}
w = model.SVs' * model.sv_coef;
b = -model.rho;
if model.Label(1) == -1
  w = -w;
  b = -b;
end
hold on 
sv = full(model.SVs);
plot(sv(:,1),sv(:,2),'kO', 'MarkerSize', 10);

y_hat = sign(w'*X' + b);
plot_x = linspace(min(X(:,1)), max(X(:,1)), 30);
plot_y = (-1/w(2))*(w(1)*plot_x + b);
d_plus = plot_y + 1/norm(w); 
d_minus = plot_y - 1/norm(w);
plot(plot_x, d_plus, 'r--'); 
hold on; 
plot(plot_x, d_minus, 'b--'); 
plot(plot_x, plot_y, 'k-', 'LineWidth', 1)
title('Plot of shattering hyperplane and support vectors'); 
legend('+1', '-1', 'Support Vectors', 'margin', '-margin', 'Hyperplane', 'Location', 'Northwest')
saveas(gcf,'tmp/svmplot.eps','epsc')

%{
\end{matlab}

\begin{figure}[h]
\centering
\includegraphics[width=4in]{tmp/svmplot.eps}
\caption{Plot of shattering hyperplane and support vectors}
\label{fig:svm} 
\end{figure}

\begin{lstlisting}
%}
delete 'tmp/lsqlin.txt'
diary 'tmp/lsqlin.txt'
z = lsqlin(X, Y)  
diary off
%{
\end{lstlisting}

\color{lightgray}
\VerbatimInput{tmp/lsqlin.txt}
\color{black}

As with the SVM, the whole point of the least squares classifier
is to solve the equation: 
\begin{equation}
f(x) = \hat{y}_n = \bm{w}^T\bm{x}_n + b
\end{equation}

So now we can plot the least squares result and compare it to the SVM
result. The results of the figure are show in Figure \ref{fig:lsq}. 
\begin{lstlisting}
%}
y_lsq = -(z(1)/z(2))*plot_x; 
plot(plot_x, y_lsq); 
legend('+1', '-1', 'Support Vectors', 'margin', '-margin', 'Hyperplane',...
   'Least Squares', 'Location', 'Northwest')
saveas(gcf,'tmp/lsqplot.eps','epsc')
%{
\end{lstlisting}

\begin{figure}[h]
\centering
\includegraphics[width=4in]{tmp/lsqplot.eps}
\caption{Plot of shattering hyperplane, support vectors and the least
squares solution}
\label{fig:lsq} 
\end{figure}

We can see from Figure \ref{fig:lsq} that almost all the same points are 
classified except for two: it properly classifies one $-1$ value and 
misclassifies one $+1$ value. 

\section{Estimating The Structural Risk}
\begin{matlab}
%}
% Set the random seed for reproducibility 
rand('seed', 1e5);
randn('seed', 1e5);
close all 
clear all

%{
\end{matlab}
The goal of this section is plot the structural risk ($R(\alpha)$) and 
the empirical risk ($R_{emp}(\alpha)$). The function that is used for
generating the points is as follows (this function was given to us 
in the homework): 

\lstinputlisting{DataGenerator.m}

Given that function we then do the following
\begin{itemize}
\item Set C to a value: 
\begin{lstlisting}
%}
C = 100; 
%{
\end{lstlisting}
\item Generate a set of 100 data, which will have a noise standard
deviation equal to $\sigma/2$. The data generator function below generates
data that is 10 dimensions. There are eight points describing a cube 
of dimension $\sigma$, four points are at one side of the planed described by
Equation \ref{eq:cube}. Here the $b=0$, at a distance $\sigma/2$ and they are
labelled with $-1$. The other four are on the other side of the plane and are labelled
as $+1$. 

\begin{equation}
\Pi \equiv \mathbf{w}^T\mathbf{x} + b , \mathbf{w}_i = 1/\sqrt(10)
\label{eq:cube}
\end{equation}
 
\begin{lstlisting}
%}
sigma = .2; 
N = 100; 
[X1, Y1] = DataGenerator(N, sigma/2); 
%{
\end{lstlisting}  

\item Train an SVM and compute its empirical error
\begin{lstlisting}
%}
delete 'tmp/train_10d.txt'
diary 'tmp/train_10d.txt'
params = sprintf('-s 0 -t 0 -c %d', C); 
m1 =  svmtrain(Y1, X1, params);
diary off; 
%{
\end{lstlisting} 

\color{lightgray}
\VerbatimInput{tmp/train_10d.txt}
\color{black}

Now that we have trained the svm, we need to compute the empirical error (also known as the
empirical risk). This is defined below in Equation \ref{eq:emp_risk}. 

\begin{equation}
R_{emp}(\alpha) = \frac{1}{2l} \sum\limits_{i=1}^l|y_i - f(x_i, \alpha)|
\label{eq:emp_risk}
\end{equation}

In Equation \ref{eq:emp_risk},
\begin{itemize}
\item $l$ - The number of observations
\item $y_i$ - is the the truth, in the case of the code this represents 
\texttt{Y1}
\item $\alpha$ - These are the support vector coefficients from the
training. 
\item $x_i$ - These are the observations, in the case of the code, this is
related to variable \texttt{X1}. 
\end{itemize}

So we can now perform the calculation

\begin{lstlisting}
%}
l = size(Y1,1); 
loss = 0; 
% Only certain observations contain non-zero values, hence we only index
% into the vectors calculated from training. 
for i = 1:size(m1.sv_indices,1); 
   loss = loss + abs(Y1[m1.sv_indices[i] - 
end
%{
\end{lstlisting}

\end{itemize}
\end{document}  
%}


